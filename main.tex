\documentclass[a4paper, 11pt]{article}
\usepackage{comment} % enables the use of multi-line comments (\ifx \fi)
\usepackage{lipsum} %This package just generates Lorem Ipsum filler text.
\usepackage{fullpage} % changes the margin
\usepackage{caption}
\usepackage{graphicx}
\usepackage[ruled]{algorithm2e}
\begin{document}
% Header-Make sure you update this information!!!!
\noindent
\large\textbf{Assignment 1 Report} \hfill \textbf{Yunjie Cao and Minjia Jin}\\
%\normalsize ECE 100-003 \hfill Teammates: Student1, Student2 \\
%Prof. Oruklu \hfill Lab Date: XX/XX/XX \\
%TA: Adam Sumner \hfill Due Date: XX/XX/XX

\large\textbf{Search: The Reversi Game}

\section*{Introduction}
This program allows the user to play a full Othello game against a computer. The computer's moves are made according the search result, which is obtained using the Minimax algorithm together with Alpha-Beta pruning. The person's moves are provided to the program through keyboard input. We can select for the computer the colour it will play and set a time limit for its response. When it is the person's turn to move, the program will list all the legal moves for the person. After every move, the program will show the current state of this game. In the end, the program will announce the winner and exit gracefully.

\section*{Program Input}
Upon running this program, the user selects the colour the computer should play as. This is done by entering a number: 1 means the computer plays as the white player and 2 means it is the dark player. After that, the user defines the time limit within which the computer should respond. The input is an integer and it determines the search depth.
\\
\\
Now, the Othello game starts. In every round, the user's move is provided to the program using the keyboard, where the input is the standard move notation, i.e. "X Y" where X is one of a-h (column) and Y is one of 1-8 (row). For example, "a 5" is a legal input. Note that there is a space between the character (a-h) and the integer (1-8).

\section*{Program Output}
There are two parts to the output of the program.
\\
\\
First, before the user makes a move, the program will present all the possible moves for the user.
\\\\
Second, after every move, the current state of the game will be shown in the form of a grid. The symbols used are '\#', which means blank, 'd', which means a dark piece, and 'w', which means a white piece. An example of the game state visualisation is shown in Fig \ref{fig:one}.

\begin{figure}[!htbp]
  \includegraphics[scale=0.5]{output.png}
  \caption{Output example}
  \label{fig:one}
\end{figure}

\section*{Design and Implementation}
\subsection{Algorithm and Evaluation Function}
In this program, the computer's optimal move is calculated by the search algorithm. Minimax adversarial search algorithm with Alpha-Beta pruning introduced is used to generate the computer's moves. When it is the computer's turn to move, it will iterate through all its possible moves and use the search algorithm to obtain the heuristic value of this move. It will then choose the move that gives it the best value - the largest value obtained - as its next move. The evaluation function is defined as the following:
\begin{equation}
\label{eqn:09}
V(node) = N_{computer} - N_{person}
\end{equation}

$N_{computer}$ means the number of pieces held by the computer and $N_{person}$ means the number of pieces held by the person.

\subsection{Implementation}
Assume that the computer is the white player and the person is the dark player. As the dark player starts first, the person moves first and the computer moves next. The implementation of this program can be shown in algorithm~\ref{alg:one}
\begin{algorithm}[!htbp]
\SetAlgoNoLine
\Repeat{End of the game}{
GenerateLegalMoves(); //generate possible moves for the person\\
PersonInput();  //person inputs his move\\
ChangeBoard();  //update the board according to person's move\\
Visualize();    //visualise the current state of the board\\
ComputerMove(); //search for a move for the computer\\
ChangeBoard();  //update the board according to computer's move\\
Visualize();    //visualise the current state of the board
}
\caption{Game procedure}
\label{alg:one}
\end{algorithm}

The most important elements used in best computer move calculation are implemented in the function $ComputerMove()$. When the computer starts searching for the optimal move, it is the MaxPlayer and his child nodes are MinPlayers. This is how the Min-max algorithm used in this program. Alpha-Beta pruning is also introduced to improve the efficiency of the search and a depth limit is set according to the predefined time limit to limit the depth of search. \\

Details of this program can be seen from the Assignment1.cpp file, which is in the directory: /h/d3/x/mi3774ji-s/Documents/AppliedAI/assignment1. Using the command ./Assignment1, this program can be run and tested in the case that the input is as described in the Program Input section and all the inputs are valid, i.e. the character is between a-h and it is lowercase.


\end{document}
